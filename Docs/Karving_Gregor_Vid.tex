\documentclass{article}
\usepackage[utf8]{inputenc}
\usepackage{amsmath}
\usepackage{graphicx} % Required for inserting images
\usepackage[slovene]{babel}
\usepackage[a4paper,top=2cm,bottom=2cm,left=3cm,right=3cm,marginparwidth=1.75cm]{geometry}


\usepackage{amsmath}
\usepackage{graphicx}
\usepackage[colorlinks=true, allcolors=blue]{hyperref}

\title{Carving}
\author{Kunc Gregor, Jerič Vid}

\begin{document}
\maketitle

\section{Zgodovina Carvinga}

\subsection*{Razvoj tehnike}
Carving je tehnika ki je bila razvita na začetku 20. stoletja. Izumitelj te tehnike je bil Francoz "Georges Joubert", ki je uporabljal smuči s povečanim stranskim lokom.
Pri carvingu gre za tehniko pri kateri se zavija brez stranskega oddrsavanja smuči "Driftanja smuči".
Smučka je za tiste čase zelo nenavadna saj je na sprednjem delu zelo široka, na zadnjem pa zelo ozka.
Cilj pri karvingu je da se smučka med zavijem zareže v sneg po celi dolžini smučke. Smučarji so pri uporabi te tehnike zelo hitrejši, zavoji so pa daljši in bolj krožni. 
Pri tem morajo smučarji paziti da sta obe nogi med zavojem enako obremenjeni. 

\subsection*{Razvoj materialov}
Carving smuči so skozi svoje obdobje doživele veliko sprememb v obliki in materialih.
Ta so smučarjem omogočale bodisi boljšo varnost, zmogljivost, udobje itd..
Ob zgodnjem času carvinga so bile smučke bolj ali manj izdelane predvsem iz lesa.
Te so bile težke, manj odzivne in težko so se prilagajale raznim razmeram okolice oz. smučišča.
S časoma so se pojavili tudi kompoziti s steklenimi vlaknami ter raznimi kovinami.
Te so smučkam izboljšale lasnosti kot so trdota in odzivnost.


Med 80-im desetletjem 19. stoletja so se na trgu začele pojavljati nove plastične carving smuči.
Te so od prejšnih ponujale več prilagodljivosti. Poleg tega so bile cenejše za izdelavo saj je bila plastika boljša za oblikovanje kot les.
Smuči so bile tudi marginalno lažje od predhodnih lesenih smuči.


Takoj po 80-tem deseletju 19 stoletja so se pojavile nove oblike carving smuči. Te so bile parabolične oblike. Tukaj se materiali niso kaj preveč spremenili
spremenila pa se je oblika smuči, kar je omogočalo boljše in lažje polaganje ovinkov.


Razvoj smuči se je potem kasneje bolj fokusiral na kombinacijo materialov.
Kombinacijski materiali ki v velikem deležu prisotnu še danes so: 
\begin{itemize}
    \item Karbonska vlakna
    \item Aluminij
    \item Titan
    \item Razni kompotizi lesa, plastike itd..
\end{itemize}
Te materiali so še izboljšali odzivnost smučk ter zmanjšale izgubo hitrosti pri smučarjih.

\*

Poleg tega da imamo razne materiale iz katerih so carving smučke izdelane, je pomembna tudi obdelava smuči.
Ta obdelava vključuje razne obloge in premaze. Te so smučkam postale pomembne saj so izboljšale drsenje ter zmanjšale trenje. 
Ta so se skozi leta tudi razvila in omogočila boljše lasnosti.


\subsection*{Tehnologija}
Razvoj tehnologije pri carving smučeh se je skozi leta razvijal paralelno z izborom materialov.
Neke ključne tehnološke inovacije skozi leta so:
\begin{itemize}
    \item Parabolična oblika smuči
    \item Napredna konstrukcija
    \item Sodobni "inteligentni" materiali
    \item Amortizacija in blaženje
    \item Razvoj profilov smuči
    \item Uporaba računalniškega modeliranja
\end{itemize}

\*

Razvoj tehnologije smuči skozi leta je zaznamovan z inovacijami v materialih, oblikovanju smuči, konstrukciji smuči, kar omogoča smučarjem izkušnjo, ki je bolj prilagojena njihovim željam in sposobnostim.

\subsection*{Infrastruktura}
Z vzponom in razvojem karvinga je teno povezana tudi infrastruktura ki jo spremlja.
Carving smuči zajema zelo različna področja infrastrukture. Zajema vse od smučarskih prog in centrov, območij za testiranje fizikalnih lasnosti smuči, trgovin ter samih tovarn kjer same smuči izdelajo.

Smučarski centri in proge so zelo tesno povezane saj so ključna destinacija, kjer smučarji izvajajo tehniko carvinga.
Razvite so bile posebne proge katere so prilagojene za carving. Te proge vključujejo primerno oblikovanje zavojev izključno za carving.
Smučarski centri so pogosto v ospredju, ko gre za preizkušanje in izboljševanje smuči.

Laboratoriji oz. inovacijski centri so pomembni saj se proizvajalci smuči pogosto zanašajo na sam razvoj smuči.
Zelijo imeti čim cenejši ter boljši izdelek na trgu. Ti centri so ključen del izboljšav karvinga skozi njegovo življensko obdobje.

Trgovine so pomemben del povezane infrastrukture, saj smučarjem omogočijo dostop do najnovejše in najnaprednejše opreme, vključno s carving smučmi. 
Poleg tega osebje v trgovinah lahko pomagajo pri izbiri karving smuči, katere bi bile najbolje primerne za kupca.

Poleg tega je pomemben del infrastrukture tudi virtualna infrastruktura kot so na primer oglasi ali spretne informacije oz. spletni članki testov smuči.
Ta ponuja novemu kupcu opcijo da si lahko sam izbere smuči po njegovih potrebah oz. finančnih zmožnostih.

\end{document}